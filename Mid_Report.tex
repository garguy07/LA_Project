\documentclass{article}
\usepackage{graphicx} % Required for inserting images
\usepackage{amsmath}
\usepackage{hyperref}

\title{\textbf{Centrality Metrics in Graph Node Networks}}

\author{By: Gargie Tambe, Asish Bharadwaj, Chetan Vellanki}

\date{June 2023}

\begin{document}

\maketitle

\begin{center}
    \Large \textbf{Interim Report}
\end{center}

\section*{Introduction}
Centrality measures play a critical role in the field of graph analytics as they provide valuable insights into the importance and influence of individual nodes within a network. By quantifying the centrality or "centrality" of nodes, researchers can identify key entities that have a significant impact on network dynamics, control the flow of information, or act as vital connectors within the network structure.

\begin{center}
    \includegraphics[scale=0.25]{graph.png}  
\end{center}

The concept of centrality encompasses multiple perspectives, allowing researchers to evaluate the significance of nodes from various angles. This versatility is particularly valuable, as it enables the analysis of complex systems such as human brain networks or social networks, where understanding the role and importance of individual nodes is crucial for comprehending the overall system dynamics.

\section*{Types of Centrality Metrics}

\begin{itemize}

\item \textbf{Degree Centrality}

Degree centrality is a fundamental measure in network analysis that quantifies the importance of a node within a graph based on its degree, which represents the number of connections it has to other nodes. It serves as a simple and intuitive indicator of node prominence, with nodes having a higher degree of centrality considered more central or influential in the network. Degree centrality is widely used in various domains, including social network analysis, transportation networks, and biological networks, providing valuable insights into key nodes and network structure. By considering the immediate connections of a node, degree centrality offers a valuable perspective on node importance without implying any negative connotations.

\item \textbf{Betweenness Centrality}

Betweenness centrality is a metric that assesses the significance of a node in a network based on its frequency in the geodesic distance or shortest path between all pairs of nodes. It identifies nodes that serve as connectors between distinct groups within the network, playing a pivotal role in controlling the flow of interactions or information. Although originating from social network analysis, betweenness centrality finds applicability in various network types, including social, transportation, and computer networks.

\item \textbf{Eigenvector Centrality}

Eigen Vector Centrality is a metric that assesses the importance of nodes in a graph based on their connections to highly important nodes. It assigns scores to nodes, considering that a node contributes more to its score if it is connected to other nodes with high scores. This measure is calculated using an iterative algorithm that converges to stable scores. Eigenvector Centrality is used in various domains, such as social network analysis, biology, and recommendation systems, to identify influential nodes within a network.

\end{itemize}

\end{document}
